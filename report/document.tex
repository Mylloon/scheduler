\documentclass{article}

\usepackage[T1]{fontenc} % encoding
\renewcommand{\familydefault}{\sfdefault} % sans-serif font

\usepackage[french]{babel} % langages
\frenchsetup{SmallCapsFigTabCaptions=false}

\newcommand{\anri}{Anri Kennel}

\newcommand{\titlename}{
  Projet : Un ordonnanceur par work stealing}
\newcommand{\mytitle}{\href{https://www.irif.fr/~jch/enseignement/systeme/projet.pdf}{\titlename}}

\usepackage[
  pdfauthor={\anri},     % author metadata
  pdftitle={\titlename}, % title  metadata
  hidelinks,             % clickable links in table of contents
]{hyperref}

% Add \extra info to title
\makeatletter
\providecommand{\extra}[1]{
  \apptocmd{\@author}{
    \end{tabular}
    \par\vspace*{0.7em}
    \begin{tabular}[t]{c}
    #1}{}{}
}
\makeatother


\title{\mytitle}
\author{\anri\thanks{\anri : 22302653}}
\extra{Programmation système avancée $\cdot$ Université Paris Cité}
\date{Année universitaire 2023-2024}


\begin{document}
\maketitle
\tableofcontents
\clearpage

\section{Description}
\dots

% \subsection[Pile VS File]{Choix d'une pile par rapport à une file}
% Vu qu'on "vole" des tâches aux autres threads, on récupère directement
% ce que l'on veut dans une pile au lieu de devoir faire des calculs en plus
% pour une file.
%
% De plus, la tâche la plus récente est celle qui est en haut de la pile, c'est
% ce que l'on veut récupérer.
%
% En résumé, une pile est préférée à une file dans un ordonnanceur en raison de
% sa simplicité, de son efficacité dans la gestion de la charge de travail.


\section{Statistiques}
\dots

\end{document}
